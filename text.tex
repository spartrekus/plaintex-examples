
\beginsection 1. Introduction.

This is the start of the introduction.
The reconstruction conjecture states that the multiset of unlabeled
vertex-deleted subgraphs of a graph determines the graph, provided it
has at least 3 vertices.  A version of the problem was first stated
by Stanis\l aw Ulam.  In this paper, we show that the conjecture can
be proved by elementary methods.  It is only necessary to integrate
the Lenkle potential of the Broglington manifold over the quantum
supervacillatory measure in order to reduce the set of possible
counterexamples to a small number (less than a trillion).  A simple
computer program that implements Pipletti's classification theorem
for torsion-free Aramaic groups with simplectic socles can then
finish the remaining cases.

\bigskip

It is possible to write text directly.

{\bf bold} is bold.
{\bf\it bold-italic} is bold and italic.

Line of text, 
just another line of text,
and another line of text.

The reconstruction conjecture states that the multiset of unlabeled
vertex-deleted subgraphs of a graph determines the graph, provided it
has at least 3 vertices.  A version of the problem was first stated
by Stanis\l aw Ulam.  In this paper, we show that the conjecture can
be proved by elementary methods.  It is only necessary to integrate
the Lenkle potential of the Broglington manifold over the quantum
supervacillatory measure in order to reduce the set of possible
counterexamples to a small number (less than a trillion).  A simple
computer program that implements Pipletti's classification theorem
for torsion-free Aramaic groups with simplectic socles can then
finish the remaining cases.

The reconstruction conjecture states that the multiset of unlabeled
vertex-deleted subgraphs of a graph determines the graph, provided it
has at least 3 vertices.  A version of the problem was first stated
by Stanis\l aw Ulam.  In this paper, we show that the conjecture can
be proved by elementary methods.  It is only necessary to integrate
the Lenkle potential of the Broglington manifold over the quantum
supervacillatory measure in order to reduce the set of possible
counterexamples to a small number (less than a trillion).  A simple
computer program that implements Pipletti's classification theorem
for torsion-free Aramaic groups with simplectic socles can then
finish the remaining cases.

\beginsection 2. Methods

This section is for the description of methods, which were used in this manuscript.
A simple computer program that allows produce dvi is here needed.

The reconstruction conjecture states that the multiset of unlabeled
vertex-deleted subgraphs of a graph determines the graph, provided it
has at least 3 vertices.  A version of the problem was first stated
by Stanis\l aw Ulam.  In this paper, we show that the conjecture can
be proved by elementary methods.  It is only necessary to integrate
the Lenkle potential of the Broglington manifold over the quantum
supervacillatory measure in order to reduce the set of possible
counterexamples to a small number (less than a trillion).  A simple
computer program that implements Pipletti's classification theorem
for torsion-free Aramaic groups with simplectic socles can then
finish the remaining cases.

The reconstruction conjecture states that the multiset of unlabeled
vertex-deleted subgraphs of a graph determines the graph, provided it
has at least 3 vertices.  A version of the problem was first stated
by Stanis\l aw Ulam.  In this paper, we show that the conjecture can
be proved by elementary methods.  It is only necessary to integrate
the Lenkle potential of the Broglington manifold over the quantum
supervacillatory measure in order to reduce the set of possible
counterexamples to a small number (less than a trillion).  A simple
computer program that implements Pipletti's classification theorem
for torsion-free Aramaic groups with simplectic socles can then
finish the remaining cases.

\bye
